\documentclass[a4paper,11pt]{article}
\usepackage{a4wide}
\usepackage{amsmath}
\usepackage{amssymb}

\begin{document}
\begin{center}
  {\LARGE\bf Supplemental note for Week 3 Part 1}
  \end{center}
\begin{flushright}
  {\large\bf ver. 20170422-01}\\
 \ \\
{\large\bf Ryoichi Yamamoto}\\
\end{flushright}

\section{Vector products}

For $\mathbf{A}=(A_x, A_y,A_z)$ and  $\mathbf{B}=(B_x, B_y,B_z)$

\subsection{Dot product (=scalar)}

$$
\mathbf{A}\cdot\mathbf{B}=A_xB_x+A_yB_y+A_zB_z=S
$$

Equivalent Einstein convention: $S=A_\alpha B_\alpha$

\subsection{Cross product (=vector)}

$$
\mathbf{A}\times\mathbf{B}=(
      A_yB_z -  A_zB_y ,
      A_xB_z -  A_zB_x ,
      A_xB_y -  A_yB_x 
)
=(V_x,V_y,V_z)
$$

Equivalent Einstein convention: $V_\alpha=\epsilon_{\alpha,\beta,\gamma}A_\beta B_\gamma$, where $\epsilon_{\alpha,\beta,\gamma}$ is the Levi-Civita Symbol
$$
\epsilon_{\alpha,\beta,\gamma}=
\begin{cases}
    +1\ \ \ \ \ {\rm if\ ({\it \alpha,\beta,\gamma})=(x,y,z),\ (y,z,x),\ or\ (z,x,y)}\\
    -1 \ \ \ \ \ {\rm if\ ({\it \alpha,\beta,\gamma})=(z,y,x),\ (y,x,z),\ or\ (x,z,y)}\\
    0\ \ \ \ \ \ \ \ {\rm if}\ \alpha=\beta,\ \alpha=\gamma,\ {\rm or}\ \beta=\gamma
\end{cases}
$$

\subsection{Dyadic product (=tensor)}

$$
\mathbf{A}\mathbf{B}=\left(
\begin{array}{ccc}
      A_xB_x & A_xB_y & A_xB_z \\
      A_yB_x & A_yB_y & A_yB_z \\
      A_zB_x & A_zB_y & A_zB_z
\end{array}\right)
=\left(
\begin{array}{ccc}
      M_{xx} & M_{xy} & M_{xz} \\
      M_{yx} & M_{yy} & M_{yz} \\
      M_{zx} & M_{zy} & M_{zz}
\end{array}\right)
$$

Equivalent Einstein convention: $M_{\alpha\beta}=A_\alpha B_\beta$


\end{document}


Let us start with the Langevin equation
\begin{equation}
m\frac{d\mathbf{V}(t)}{dt}={-\zeta\mathbf{V}(t)}+{\mathbf{F}(t)},
\tag{F2}
\end{equation}
where the random force $\mathbf{F}(t)$ satisfies the following conditions
\begin{equation}
\langle \mathbf{F}(t)\rangle=\mathbf{0}
\tag{F3}
\end{equation}
\begin{equation}
\langle \mathbf{F}(t)\mathbf{F}(0)\rangle = {2k_B T\zeta}\mathbf{I}\delta(t),\tag{F4}
\end{equation}
with
$\mathbf{0}\equiv(0,0,0)$ and $\mathbf{I}\equiv\begin{bmatrix}1&0&0\\0&1&0\\0&0&1\end{bmatrix}$.\\
We now discretize the time $t$ using an increment $\Delta t$, such that
$t_i\equiv i\Delta t$, and define the cumulative impulse during the
interval $t_i\le t\le t_{i+1} = t_i + \Delta t$, as
\begin{equation}
\Delta\mathbf{W}_i
\equiv\int_{t_i}^{t_{i+1}} dt\mathbf{F}(t).
\tag{F8}
\end{equation}
From Eq.(F3), it is straightforward to show
\begin{equation}
\langle\Delta\mathbf{W}_i\rangle
=\int_{t_i}^{t_{i+1}} dt\langle\mathbf{F}(t)\rangle=\mathbf{0}.
\tag{F10}
\end{equation}
Also from Eq.(F4), for $j\ne i$
\begin{eqnarray}
\langle\Delta\mathbf{W}_i\Delta\mathbf{W}_{j\ne i}\rangle
&=&\int_{t_i}^{t_{i+1}} dt\int_{t_j}^{t_{j+1}} dt'\langle\mathbf{F}(t)\mathbf{F}(t')\rangle\\
&=&2k_B T\zeta\mathbf{I} \int_{t_i}^{t_{i+1}} dt\int_{t_j}^{t_{j+1}} dt'
\delta(t-t')\\
&=&\mathbf{O},
\end{eqnarray}
where $\mathbf{O}\equiv\begin{bmatrix}0&0&0\\0&0&0\\0&0&0\end{bmatrix}$.\\
For $j=i$
\begin{eqnarray}
\langle\Delta\mathbf{W}_i\Delta\mathbf{W}_i\rangle
&=&\int_{t_i}^{t_{i+1}} dt\int_{t_i}^{t_{i+1}} dt'\langle\mathbf{F}(t)\mathbf{F}(t')\rangle\\
&=&2k_B T\zeta\mathbf{I} \int_{t_i}^{t_{i+1}} dt\int_{t_i}^{t_{i+1}} dt'
\delta(t-t')\\
&=&2k_B T\zeta\Delta t\mathbf{I}.
\end{eqnarray}
Combining Eqs.(3) and (6), we obtain
\begin{equation}
\langle \Delta \mathbf{W}_i\Delta \mathbf{W}_j\rangle = {2k_B T\zeta}\Delta t\mathbf{I}\delta_{ij} .
\tag{F11}
\end{equation}

%\newpage
\section{Distribution of $\Delta\mathbf{W}_i$}

Here we further divide $\Delta t$ into $n$ segments ($n\gg1$) of a
very small time span $\epsilon$, {\it i.e.}, $\Delta t \equiv n\epsilon$, and 
define a new cumulative impulse over $\epsilon$
\begin{equation}
\mathbf{W}^m_i
\equiv\int_{t_i+(m-1)\epsilon}^{t_{i}+m\epsilon} dt\mathbf{F}(t),
\end{equation}
where $1\le m\le n$.\\
Repeating the same procedure performed in the previous section, the following conditions are derived.
\begin{eqnarray}
\langle\mathbf{W}^m_i\rangle&=&\mathbf{0}\\
\langle \mathbf{W}^m_i\mathbf{W}^l_j\rangle &=& {2k_B T\zeta}\epsilon\mathbf{I}\delta_{ij}\delta_{ml}
\end{eqnarray}
Eqs.(8) and (9) show that the mean and variance of the random numbers $W^m_{\alpha,i}$ $(\alpha\in x,y,z)$ are zero and $2k_B T\zeta\epsilon$, respectively. \\
From Eq.(F8) and (7), we should notice that
\begin{equation}
\Delta\mathbf{W}_i
=\mathbf{W}^1_i+\mathbf{W}^2_i+\cdots+\mathbf{W}^n_i.
\end{equation}
Therefore, from the central limit theorem Eqs.(D7)-(D9) introduced in
Part 3 of Week 2, one realizes that the $\Delta W_{\alpha,i}$ should be
drawn from a \textit{Gaussian} distribution, with average and variance
equal to zero and $2k_B T\zeta\Delta t$, respectively, regardless of the
distribution of the $W^m_{\alpha,i}$.
\end{document}

