\documentclass[a4paper,11pt]{article}
\usepackage{a4wide}
\usepackage{amsmath}
\usepackage{amssymb}
\begin{document}
\begin{center}
  {\LARGE\bf Supplemental note}
  \end{center}
\begin{flushright}
  {\large\bf ver. 20170209-01}\\
 \ \\
{\large\bf Ryoichi Yamamoto}\\
\end{flushright}
\section{Binomial distribution $\rightarrow$ Normal distribution}

Start from Eq.(C6) shown below.
\begin{eqnarray}
  P(n)&=&\frac{M!}{n!(M-n)!}p^n(1-p)^{M-n}\\
  \ln P(n)&=&\ln M! -\ln n! - \ln (M-n)!+n\ln p +(M-n)\ln(1-p)\\
  &=&M(\ln M -1) -n(\ln n -1) - (M-n)(\ln (M-n)-1)\\& & + n\ln p +(M-n)\ln (1-p)
\end{eqnarray}
Here we used Stirling's approximation  valid for large $n$.
\begin{equation}
\ln n!\simeq n(\ln n -1)
\end{equation}
Take the 1st derivative of the above equation in terms of $n$.
\begin{eqnarray}
  \frac{d \ln P}{dn}&=&-(\ln n +1) + (\ln(M-n)+1) + \ln p -ln(1-p)\\
  &=&-\ln n + \ln(M-n)+\ln p -\ln(1-p)\\
  &=&\ln \left[\frac{M-n}{n} \right] - \ln \left[\frac{1-p}{p} \right] 
\end{eqnarray}
When $n$ and $M$ are both large, the peak in $P(n)$ and also in $\ln P(n)$ is very sharp around the mean value $n=\langle N\rangle$, and thus
\begin{eqnarray}
  \left.\frac{d \ln P}{dn}\right|_{n=\langle N\rangle} &=&\ln \left[\frac{M-\langle N\rangle}{\langle N\rangle} \right] - \ln \left[\frac{1-p}{p} \right] =0\\
  \therefore \frac{M-\langle N\rangle}{\langle N\rangle} &=& \frac{1-p}{p} \\
  \langle N\rangle&=&Mp
\end{eqnarray}


Consider a Taylor expansion of $\ln P(n)$ around the mean $\langle N\rangle$ by defining $n=\langle N\rangle+\delta n$.
\begin{equation}
  \ln P(n) = \ln (\langle N\rangle) + \left.\frac{d \ln P}{dn}\right|_{n=\langle N\rangle}\delta n
  + \frac{1}{2}\left.\frac{d^2 \ln P}{dn^2}\right|_{n=\langle N\rangle}\delta n^2 + \cdots
  \end{equation}
From Eq.(7),
\begin{eqnarray}
  \frac{d^2 \ln P}{dn^2}&=&-\frac{1}{n}-\frac{1}{M-n}\\
  \left.\frac{d^2 \ln P}{dn^2}\right|_{n=\langle N\rangle}&=&-\frac{1}{Mp}-\frac{1}{M-Mp}=-\frac{1}{Mp(1-p)}=-\sigma^{-2}
\end{eqnarray}
Terminate Eq.(12) at the 2nd order in terms of  $\delta n$, and use Eqs.(9) and (14). 
\begin{equation}
P(n)=const.\times\exp\left(-\frac{\delta n^2}{2\sigma^2}\right)
\end{equation}
Determine $const.$ so that that $\int_{-\infty}^{\infty}P(n)dn=1$, we finally obtain the following normal distribution function with $\langle N\rangle=Mp$ and $\sigma^2=Mp(1-p)$.
\begin{equation}
P(n)=\frac{1}{\sqrt{2\pi\sigma^2}}\exp\left(-\frac{(n-\langle N\rangle)^2}{2\sigma^2}\right)
\end{equation}

\section{Binomial distribution $\rightarrow$ Poisson distribution}

We now consider the limit of $M\rightarrow\infty$ while $\langle N\rangle=Mp=a$ remains constant.
Notice that the following approximations hold.
\begin{eqnarray}
%p=\frac{a}{M-1}&\simeq&\frac{a}{M}\\
M-n&\simeq& M\\
  \frac{M!}{(M-n)!}&=&M(M-1)\cdots(M-n+1)\simeq M^n
\end{eqnarray}
Again start from Eq.(C6), we can derive Poisson distribution as shown below
with $\langle N\rangle=Mp=a$ and $\sigma^2=Mp(1-p)\simeq a$.
\begin{eqnarray}
  P(n)&=&\frac{M!}{n!(M-n)!}p^n(1-p)^{M-n}\\
  &\simeq&\frac{1}{n!}M^n\left(\frac{a}{M}\right)^n\left(1-\frac{a}{M}\right)^M\\
  &\simeq&\frac{a^n e^{-a}}{n!}
\end{eqnarray}
Here we used
\begin{equation}
\lim_{M\rightarrow \infty}\left(1-\frac{a}{M}\right)^M= e^{-a}
\end{equation}
\end{document}

